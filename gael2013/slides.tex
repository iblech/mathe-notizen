\documentclass[10pt,utf8,notheorems,compress]{beamer}

\usepackage[english]{babel}

\usepackage{amsmath,amssymb}
%\usepackage[framed,amsmath,thmmarks,hyperref]{ntheorem}

%\usepackage[small,nohug]{diagrams}
%\diagramstyle[labelstyle=\scriptstyle]

%\usepackage[protrusion=true,expansion=false]{microtype}

%\usepackage{lmodern}
\usepackage{tabto}
%\usepackage{tikz}
\usepackage{array}
%\usepackage[all]{xy}

\setlength\parskip{\medskipamount}
\setlength\parindent{0pt}

%\theoremseparator{:}
\theoremstyle{plain}  %nonumberplain
%\newtheorem{beh}{Behauptung}
\newtheorem{proposition}{Proposition}
\newtheorem{corollary}{Korollar}
\newtheorem{theorem}{Satz}
\theoremstyle{definition}
\newtheorem{definition}{Definition}
\newtheorem{baddefinition}{Misleading definition}
\newtheorem{motto}{Motto}
%\newtheorem{kor}{Korollar}
%\newtheorem{satz}{Satz}
%\newtheorem{lemma}{Lemma}
%\newtheorem{hilfsaussage}{Hilfsaussage}
%\theorembodyfont{\normalfont}
\newtheorem{axiom}{Axiom}
%\newtheorem{defnprop}{Definition/Proposition}
%\newtheorem{bem}{Bemerkung}
\newtheorem{examples}{Examples}
%\theoremsymbol{\ensuremath{\openbox}}
%\newtheorem{proof}{Beweis}
%\newtheorem{defn}{Definition}

\newcommand{\lra}{\longrightarrow}
\newcommand{\lhra}{\ensuremath{\lhook\joinrel\relbar\joinrel\rightarrow}}
\newcommand{\thlra}{\relbar\joinrel\twoheadrightarrow}

\newcommand{\ZZ}{\mathbb{Z}}
\renewcommand{\C}{\mathcal{C}}
\newcommand{\D}{\mathcal{D}}
\newcommand{\E}{\mathcal{E}}
\newcommand{\F}{\mathcal{F}}
\renewcommand{\O}{\mathcal{O}}
\newcommand{\NN}{\mathbb{N}}
\newcommand{\RR}{\mathbb{R}}
\newcommand{\Hom}{\mathrm{Hom}}
\newcommand{\id}{\mathrm{id}}
\newcommand{\Aut}[1]{\operatorname{Aut}(#1)}
\newcommand{\GL}[1]{\operatorname{GL}(#1)}
\newcommand{\freist}{\_{}\_{}}
\newcommand{\Set}{\mathrm{Set}}
\newcommand{\Grp}{\mathrm{Grp}}
\newcommand{\Vect}{\mathrm{Vect}}
\newcommand{\Sh}{\mathrm{Sh}}
\newcommand{\Zar}{\mathrm{Zar}}
\newcommand{\?}{\,{:}\,}
\renewcommand{\_}{\mathpunct{.}\,}
\newcommand{\speak}[1]{\ulcorner\text{#1}\urcorner}

\def\longleadsto{\mathrel{-}\joinrel\leadsto}
\DeclareMathOperator{\ggT}{ggT}
\DeclareMathOperator{\Ob}{Ob}
\newcommand{\op}{\mathrm{op}}

\title[Internal language of topoi]{Using the internal language of topoi in algebraic geometry}
\author[Ingo Blechschmidt (Univ. Augsburg)]{Ingo Blechschmidt}
\institute{University of Augsburg (Germany)}
\date{June 24th, 2013}

%\usetheme{Warsaw}  %Warsaw, Berkeley?
\usetheme{Singapore}
%\useoutertheme{split}
%\usecolortheme{seahorse}
\usefonttheme{serif}
\usepackage{palatino}
%\useinnertheme{rectangles}
%\usepackage{bookman}
%\setbeamercovered{transparent}

\setbeamertemplate{navigation symbols}{}
%\setbeamertemplate{footline}{}
%\setbeamertemplate{headline}{}

%\beamertemplateboldcenterframetitle
%\setbeamerfont{frametitle}{size={\Large}}

%\makeatletter
%\setbeamertemplate{theorem begin}
%{%
%  \inserttheoremheadfont% \bfseries
%  \inserttheoremname \inserttheoremnumber
%  \ifx\inserttheoremaddition\@empty\else\ (\inserttheoremaddition)\fi%
%  \inserttheorempunctuation
%  \normalfont
%}
%\setbeamertemplate{theorem end}{%
%  % empty
%}
%\makeatother

\newcommand*\oldmacro{}%
\let\oldmacro\insertshorttitle%
\renewcommand*\insertshorttitle{%
  \oldmacro\hfill\insertframenumber\,/\,\inserttotalframenumber\hfill}

\newenvironment{changemargin}[2]{%
  \begin{list}{}{%
    \setlength{\topsep}{0pt}%
    \setlength{\leftmargin}{#1}%
    \setlength{\rightmargin}{#2}%
    \setlength{\listparindent}{\parindent}%
    \setlength{\itemindent}{\parindent}%
    \setlength{\parsep}{\parskip}%
  }%
  \item[]}{\end{list}}

\newcommand{\slogan}[1]{%
  \begin{center}%
    \setlength{\fboxrule}{2pt}%
    \setlength{\fboxsep}{-3pt}%
    {\usebeamercolor[fg]{item}\fbox{\usebeamercolor[fg]{normal
    text}\parbox{0.9\textwidth}{\begin{center}#1\end{center}}}}%
  \end{center}%
}

\newcommand{\hil}[1]{{\usebeamercolor[fg]{item}{#1}}}

\begin{document}

\setbeameroption{show notes}
\setbeamertemplate{note page}[plain]

\frame{\titlepage}
%\frame[t]{\frametitle{Gliederung}\begin{minipage}{\textwidth}\begin{small}\tableofcontents\end{small}\end{minipage}}
\frame[t]{\frametitle{Outline}\tableofcontents}

\section{The internal language of topoi}

\frame[t]{\frametitle{What is a topos?}
  \begin{baddefinition}A \emph{topos} is a category which has finite limits,
  is cartesian closed and has a subobject classifier.
  \end{baddefinition}

  \begin{motto}A topos is a category sufficiently rich to support
  an \emph{internal language}.
  \end{motto}

  \begin{examples}\vspace{-1em}\begin{itemize}
    \item \hil{$\Set$:} \tabto{1.5cm} category of sets
    \item \hil{$\Sh(X)$:} \tabto{1.5cm} category of set-valued sheaves on a space~$X$
    \item \hil{$\Zar(S)$:} \tabto{1.5cm} gros Zariski topos of a base scheme~$S$
  \end{itemize}\end{examples}
}

\frame[t]{\frametitle{The internal language}
  
}


\section{The petit Zariski topos of a scheme}

\frame[t]{\frametitle{The petit Zariski topos}
  \begin{definition}The \emph{petit Zariski topos} of a scheme~$X$ is the
  category~$\Sh(X)$ of set-valued sheaves on~$X$.\end{definition}

  \begin{block}{Basic properties}
  \vspace{-1em}
  \begin{itemize}
    \item The structure sheaf~$\O_X$ looks like an ordinary ring.
    \item A sheaf of~$\O_X$-modules looks like an ordinary module on that ring.
  \end{itemize}
  \end{block}

}

\frame[t]{\frametitle{The structure sheaf, internally}
  \begin{itemize}
    \item Internally, $\O_X$ is a local ring: \[ \Sh(X) \models \forall x,y\?\O_X\_
      \speak{$x+y$ inv.} \Longrightarrow
      \speak{$x$ or $y$ inv.} \]

    \item Internally, $\O_X$ is ``almost'' a field:
    \[ \Sh(X) \models \forall x\?\O_X\_
      \neg\speak{$x$ inv.} \Longrightarrow \speak{$x$ is nilpotent} \]

    \item $X$ is reduced iff $\O_X$ is internally a reduced ring,
    \[ \Sh(X) \models \bigwedge_{n \geq 0} \forall x\?\O_X\_ x^n = 0 \Longrightarrow
    x = 0, \]
    iff $\O_X$ is internally a so-called residue field,
    \[ \Sh(X) \models \forall x\?\O_X\_
      \neg\speak{$x$ inv.} \Longrightarrow x = 0. \]
  \end{itemize}
}

\frame[t]{\frametitle{The sheaf of rational functions}
  \begin{itemize}
    \item Classically, one can define as the sheaf~$K_X$ of rational functions
    as the sheafification of
    \[ U \subseteq X \longmapsto \Gamma(U, \O_X)[S(U)^{-1}], \]
    where~$S(U) = \{ s \in \O_X(U) \,|\, \text{$s \in \O_{X,x}^\times$
    for all~$x \in U$} \}$.

    \item Internally, we can simply define~$K_X$ as the total quotient ring
    of~$\O_X$.
  \end{itemize}
}

\frame[t]{\frametitle{Sheaves of modules, internally}
  Let~$\F$ be a sheaf of~$\O_X$-modules.
  \begin{itemize}
    \item $\F$ is \emph{locally of finite type} iff internally, it is finitely
    generated:
    \begin{multline*}
      \Sh(X) \models \bigvee_{n\geq0} \exists x_1,\ldots,x_n\?\F\_
      \forall x\?\F\_ \exists a_1,\ldots,a_n\?\O_X\_
      x = \sum_i a_i x_i
    \end{multline*}

    \item $\F$ is \emph{locally of finite presentation} iff internally,
    it is finitely presented:
    \[ \Sh(X) \models \speak{$\exists$ surjection~$\O_X^n
    \twoheadrightarrow \F$ with f.g. kernel} \]

    \item $\F$ is \emph{locally free} iff internally, it is a free module:
    \[ \Sh(X) \models \bigvee_{n\geq0} \F \cong \O_X^n \]
  \end{itemize}
}

\frame[t]{\frametitle{Rank of a sheaf of modules}
  Let~$\F$ be an~$\O_X$-module locally of finite type. Then we have the
  upper-semicontinuous rank function
  \[ \begin{array}{@{}rcl@{}}
    X &\longrightarrow& \NN \\
    x &\longmapsto& \text{minimal number of elements needed to generate~$\F_x$.}
  \end{array} \]
  Internally, this is simply the
  \[ \text{minimal number of elements needed to generate~$\F$} \in \widehat\NN \]
  considered as an element of the completed natural numbers.

  \begin{block}{Well-known proposition}
  If~$X$ is a reduced scheme, $\F$ is locally free iff its rank is locally
  constant.\end{block}

  \begin{block}{Internal proof}
  
  \end{block}
}

\end{document}
