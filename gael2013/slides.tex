\documentclass[10pt,utf8,notheorems,compress]{beamer}

\usepackage[english]{babel}

\usepackage{amsmath,amssymb}
\usepackage{extarrows}
%\usepackage[framed,amsmath,thmmarks,hyperref]{ntheorem}

%\usepackage[small,nohug]{diagrams}
%\diagramstyle[labelstyle=\scriptstyle]

%\usepackage[protrusion=true,expansion=false]{microtype}

%\usepackage{lmodern}
\usepackage{tabto}
\usepackage{array}
%\usepackage[all]{xy}


\usepackage{tikz}
\usetikzlibrary{calc,shapes.callouts,shapes.arrows}
\newcommand{\hcancel}[5]{%
    \tikz[baseline=(tocancel.base)]{
        \node[inner sep=0pt,outer sep=0pt] (tocancel) {#1};
        \draw[red, line width=0.5mm] ($(tocancel.south west)+(#2,#3)$) -- ($(tocancel.north east)+(#4,#5)$);
    }%
}%

\setlength\parskip{\medskipamount}
\setlength\parindent{0pt}

\newcommand{\ZZ}{\mathbb{Z}}
\renewcommand{\AA}{\mathbb{A}}
\renewcommand{\C}{\mathcal{C}}
\newcommand{\D}{\mathcal{D}}
\newcommand{\E}{\mathcal{E}}
\newcommand{\F}{\mathcal{F}}
\renewcommand{\G}{\mathcal{G}}
\renewcommand{\O}{\mathcal{O}}
\renewcommand{\P}{\mathcal{P}}
\newcommand{\NN}{\mathbb{N}}
\newcommand{\RR}{\mathbb{R}}
\newcommand{\GG}{\mathbb{G}}
\newcommand{\Hom}{\mathrm{Hom}}
\newcommand{\id}{\mathrm{id}}
\newcommand{\Aut}[1]{\operatorname{Aut}(#1)}
\newcommand{\GL}{\mathrm{GL}}
\newcommand{\freist}{\_{}\_{}}
\newcommand{\Set}{\mathrm{Set}}
\newcommand{\Grp}{\mathrm{Grp}}
\newcommand{\Vect}{\mathrm{Vect}}
\newcommand{\Sh}{\mathrm{Sh}}
\newcommand{\Zar}{\mathrm{Zar}}
\newcommand{\Sch}{\mathrm{Sch}}
\DeclareMathOperator{\Spec}{Spec}
\newcommand{\?}{\,{:}\,}
\renewcommand{\_}{\mathpunct{.}\,}
\newcommand{\speak}[1]{\ulcorner\text{#1}\urcorner}
\newcommand{\Ll}{:\Longleftrightarrow}
\newcommand{\ul}[1]{\underline{#1}}
\newcommand{\affl}{\ensuremath{{\ul{\AA}^1_S}}}

\newcommand{\lra}{\longrightarrow}

\newcommand{\op}{\mathrm{op}}

\newcommand{\fmini}[2]{\framebox{\begin{minipage}{#1}#2\end{minipage}}}
\makeatletter
\def\underunbrace#1{\mathop{\vtop{\m@th\ialign{##\crcr
      $\hfil\displaystyle{#1}\hfil$\crcr\noalign{\kern3\p@\nointerlineskip}
      \crcr\noalign{\kern3\p@}}}}\limits}
\def\overunbrace#1{\mathop{\vbox{\m@th\ialign{##\crcr\noalign{\kern3\p@}
      \crcr\noalign{\kern3\p@\nointerlineskip}
      $\hfil\displaystyle{#1}\hfil$\crcr}}}\limits}
\makeatother

\newcommand{\backupstart}{
  \newcounter{framenumbervorappendix}
  \setcounter{framenumbervorappendix}{\value{framenumber}}
}
\newcommand{\backupend}{
  \addtocounter{framenumbervorappendix}{-\value{framenumber}}
  \addtocounter{framenumber}{\value{framenumbervorappendix}} 
}

\title[Using the internal language of topoi in algebraic geometry]{Using the internal language of topoi in algebraic geometry}
\author[Ingo Blechschmidt (Univ. Augsburg)]{Ingo Blechschmidt \\[0.5em]
\scriptsize
University of Augsburg (Germany)}
%\institute{--- questions welcome at all times ---}
\date{June 24th, 2013 \\[0.5em] Géométrie Algébrique en Liberté XXI}

\usetheme{Singapore}
%\useoutertheme{split}
\usefonttheme{serif}
\usepackage{palatino}
\useinnertheme{rectangles}

\setbeamertemplate{navigation symbols}{}

\setbeameroption{show notes}
\setbeamertemplate{note page}[plain]

\defbeamertemplate*{footline}{split theme}
{%
  \leavevmode%
  \hbox{\begin{beamercolorbox}[wd=.5\paperwidth,ht=2.5ex,dp=1.125ex,leftskip=.3cm plus1fill,rightskip=.3cm]{author in head/foot}%
    \usebeamerfont{author in head/foot}\insertshortauthor
  \end{beamercolorbox}%
  \begin{beamercolorbox}[wd=.5\paperwidth,ht=2.5ex,dp=1.125ex,leftskip=.3cm,rightskip=.3cm plus1fil]{title in head/foot}%
    \usebeamerfont{title in head/foot}\insertshorttitle
  \end{beamercolorbox}}%
  \vskip0pt%
}

\newcommand*\oldmacro{}%
\let\oldmacro\insertshorttitle%
\renewcommand*\insertshorttitle{%
  \oldmacro\hfill\insertframenumber\,/\,\inserttotalframenumber\hfill}

\newenvironment{changemargin}[2]{%
  \begin{list}{}{%
    \setlength{\topsep}{0pt}%
    \setlength{\leftmargin}{#1}%
    \setlength{\rightmargin}{#2}%
    \setlength{\listparindent}{\parindent}%
    \setlength{\itemindent}{\parindent}%
    \setlength{\parsep}{\parskip}%
  }%
  \item[]}{\end{list}}

\newcommand{\pointthis}[2]{%
        \tikz[remember picture,baseline]{\node[anchor=base,inner sep=0,outer sep=0]%
        (#1) {#1};\node[overlay,rectangle callout,%
        callout relative pointer={(0.2cm,0.8cm)},fill=blue!20] at ($(#1.north)+(-.5cm,-1.4cm)$) {#2};}%
        }%

\newcommand{\slogan}[1]{%
  \begin{center}%
    \setlength{\fboxrule}{2pt}%
    \setlength{\fboxsep}{-3pt}%
    {\usebeamercolor[fg]{item}\fbox{\usebeamercolor[fg]{normal
    text}\parbox{0.9\textwidth}{\begin{center}#1\end{center}}}}%
  \end{center}%
}

\newcommand{\hil}[1]{{\usebeamercolor[fg]{item}{#1}}}

\begin{document}

\setbeameroption{show notes}
\setbeamertemplate{note page}[plain]

\frame{\titlepage}
%\frame[t]{\frametitle{Gliederung}\begin{minipage}{\textwidth}\begin{small}\tableofcontents\end{small}\end{minipage}}
\frame[t]{\frametitle{Outline}\tableofcontents}

\section{The internal language of topoi}

\subsection{What is a topos?}
\frame[t]{\frametitle{What is a topos?}
  \begin{block}{Formal definition}A \hil{topos} is a category which has finite limits,
  is cartesian closed and has a subobject classifier.
  \end{block}

  \begin{block}{Motto}A topos is a category sufficiently rich to support
  an \hil{internal language}.
  \end{block}

  \begin{block}{Examples}\vspace{-0.5em}\begin{itemize}
    \item \hil{$\Set$:} \tabto{1.2cm} category of sets
    \item \hil{$\Sh(X)$:} \tabto{1.2cm} category of set-valued sheaves on a space~$X$
%   \item \hil{$\Zar(S)$:} \tabto{1.5cm} gros Zariski topos of a base scheme~$S$
  \end{itemize}\end{block}

  \note{
    \begin{itemize}
      \item While technically correct, the formal definition is actually
      misleading in a sense: A topos has lots of other vital structure, which
      is crucial for a rounded understanding, but is not listed in the
      definition, which is trimmed for minimality.

      A more comprehensive definition is: A \emph{topos} is a locally cartesian
      closed, finitely complete and cocomplete Heyting category which is exact,
      extensive and has a subobject classifier.
    \end{itemize}
  }
}

\subsection{What is the internal language?}
\frame[t]{\frametitle{What is the internal language?}
  The internal language of a topos~$\E$ allows to
  \begin{enumerate}
    \item construct objects and morphisms of the topos,
    \item formulate statements about them and
    \item prove such statements
  \end{enumerate}
  in a \hil{naive element-based} language:

  \begin{center}
    \begin{tabular}{r|l}
      external point of view & internal point of view \\\hline
      objects of~$\E$ & sets \\
      morphisms of~$\E$ & maps of sets
    \end{tabular}
  \end{center}

  \hil{Special case:} The language of~$\Set$ is the usual mathematical
  language.

  \note{
    \begin{itemize}
      \item Actually, the objects of~$\E$ feel more like \emph{types} instead of \emph{sets}:
      For instance, there is no global membership relation~$\in$. Rather,
      for each object~$A$ of~$\E$, there is a relation~${\in_A} : A \times \P(A) \to
      \Omega$, where~$\P(A)$ is the power object of~$A$.

      \item Compare with the embedding theorem for abelian categories:
      There, an explicit embedding into a category of modules is constructed.
      Here, we only change perspective and talk about the same objects and
      morphisms.

      \item There exists a weaker variant of the internal language which works
      in abelian categories. By using it, one can even pretend that the objects
      are abelian groups (instead of modules), and when constructing morphisms
      by appealing to the axiom of unique choice (which is a theorem), one
      needn't even have to check linearity. The proof that this approach
      works uses only categorical logic (so ist mostly ``just formal'').

      \item For textbook developments of the internal language,
      see~\cite{johnstone:elephant} and~\cite{moerdijk:maclane:sheaves}.
    \end{itemize}
  }
}

\frame[t]{\frametitle{The internal language of~$\Sh(X)$}
  Let~$X$ be a topological space. Then we can recursively define
  \[ U \models \varphi \quad \text{(``$\varphi$ holds on~$U$'')} \]
  for open subsets~$U \subseteq X$ and formulas~$\varphi$.
  \[ \renewcommand{\arraystretch}{1.3}\begin{array}{@{}lcl@{}}
    U \models f = g \? \F &\Ll& f|_U = g|_U \in \Gamma(U, \F) \\
    U \models \varphi \wedge \psi &\Ll&
      \text{$U \models \varphi$ and $U \models \psi$} \\
    U \models \varphi \vee \psi &\Ll&
      \hcancel{\text{$U \models \varphi$ or $U \models \psi$}}{0pt}{3pt}{0pt}{-2pt} \\
    && \text{there exists a covering $U = \bigcup_i U_i$ s.\,th. for all~$i$:} \\
    && \quad\quad \text{$U_i \models \varphi$ or $U_i \models \psi$} \\
    U \models \varphi \Rightarrow \psi &\Ll&
      \text{for all open~$V \subseteq U$: } 
    \text{$V \models \varphi$ implies $V \models \psi$} \\
    U \models \forall f \? \F\_ \varphi(f) &\Ll&
      \text{for all sections~$f \in \Gamma(V, \F), V \subseteq U$: $V \models
      \varphi(f)$} \\
    U \models \exists f \? \F\_ \varphi(f) &\Ll&
      \text{there exists a covering $U = \bigcup_i U_i$ s.\,th. for all~$i$:} \\
    && \quad\quad \text{there exists~$f_i \in \Gamma(U_i, \F)$ s.\,th.
    $U_i \models \varphi(f_i)$}
  \end{array} \]

  \note{
    \begin{itemize}
      \item The rules are called \emph{Kripke-Joyal semantics} and can be
      formulated over any topos (not just sheaf topoi). They are not all
      arbitrary: Rather, they are very finely adjusted to make the crucial
      properties about the internal language (see next slide) true.
      % XXX: make true
      % XXX: finely adjusted (meine: abgestimmt)

      \item There are two further rules concerning the constants~$\top$
      and~$\bot$ (truehood resp. falsehood):
      % XXX Begriff truehood!
        \[ \renewcommand{\arraystretch}{1.3}\begin{array}{@{}lcl@{}}
          U \models \top &\Ll& U = U \\
          U \models \bot &\Ll& U = \emptyset
        \end{array} \]

      \item Negation is defined as
      \[ \neg\varphi :\equiv (\varphi \Rightarrow \bot). \]

      \item One can extend the language to allow for \emph{unbounded}
      quantification (think~$\forall a \in A$ vs.~$\forall A$), see Shulman's
      stack semantics~\cite{shulman:stack}. This is needed to formulate
      universal properties internal to~$\Sh(X)$, for instance.

      \item One can further extend the language to be able to talk
      about locally internal categories over~$\Sh(X)$: Then one can do category
      theory internal to~$\Sh(X)$ using the internal language. (This specific
      approach is, as far as I am aware, original work. But of course, internal
      category theory has been done for a long time, see for instance the
      textbook~\cite{johnstone:elephant}; also
      cf.~\cite{chapman:rowbottom:relcat}.)
    \end{itemize}
  }
}

\note{
  \begin{itemize}
    \item Let~$\alpha : \F \to \G$ be a morphism of sheaves on~$X$. Then:
    \begin{align*}
      & X \models \speak{$\alpha$ is injective} \\[0.5em]
      \Longleftrightarrow\
      & X \models \forall s,t\?\F\_ \alpha(s) = \alpha(t) \Rightarrow s = t \\[0.5em]
      \Longleftrightarrow\ &
        \text{for all open~$U \subseteq X$, sections $s, t \in \Gamma(U, \F)$:} \\
      &\qquad\qquad
          U \models \alpha(s) = \alpha(t) \Rightarrow s = t \\[0.5em]
      \Longleftrightarrow\ &
        \text{for all open~$U \subseteq X$, sections $s, t \in \Gamma(U, \F)$:} \\
      &\qquad\qquad
          \text{for all open~$V \subseteq U$:} \\
      &\qquad\qquad\qquad\qquad
            \text{$\alpha_V(s|_V) = \alpha_V(t|_V)$ implies $s|_V = t|_V$} \\[0.5em]
      \Longleftrightarrow\ &
        \text{for all open~$U \subseteq X$, sections $s, t \in \Gamma(U, \F)$:} \\
      &\qquad\qquad
            \text{$\alpha_U(s|_U) = \alpha_U(t|_U)$ implies $s|_U = t|_U$} \\[0.5em]
      \Longleftrightarrow\ &
        \text{$\alpha$ is a monomorphism of sheaves}
    \end{align*}
  \end{itemize}
}

\frame[t]{\frametitle{The internal language of~$\Sh(X)$}
  \begin{block}{Crucial properties}
  \vspace{-1em}
  \begin{itemize}
    \item \hil{Locality:} \tabto{1.9cm} If~$U = \bigcup_i U_i$, then
      $U \models \varphi$ iff $U_i \models
      \varphi$ for each~$i$.
    \item \hil{Soundness:} \tabto{1.9cm} If~$U \models \varphi$ and $\varphi$
    implies $\psi$ \pointthis{constructively}{
      no $\varphi \vee \neg\varphi$,\quad
      no $\neg\neg\varphi \Rightarrow \varphi$,\quad
      no axiom of choice
    }, \\[0.1cm]
    \tabto{1.9cm} then~$U \models \psi$.
  \end{itemize}
  \end{block}

  \vfill

  \begin{block}{Examples}
  \vspace{-1em}
  \begin{itemize}
    \item $U \models f = 0$
      \tabto{3cm} iff $f|_U = 0 \in \Gamma(U, \F)$.
    \item $U \models f = 0 \vee g = 0$
      \tabto{3cm} iff on a cover~$U = \bigcup_i U_i$,\ \ $f|_{U_i} = 0$ or~$g|_{U_i} = 0$.
    \item $U \models \neg\neg(f = 0)$
      \tabto{3cm} iff~$f = 0$ holds on a dense open subset of~$U$.
  \end{itemize}
  \end{block}

  \note{
    \begin{itemize}
      \item The internal logic of~$\Sh(X)$ is classical (fulfills law of excluded
      middle) iff~$X$ is discrete.

      \item There is a false rumor about constructive mathematics, namely that
      the term \emph{contradiction} is generally forbidden. This is not the
      case, one has to distinguish between
      \begin{itemize}
        \item a true proof by contradiction: ``Assume~$\varphi$ were false.
        Then \ldots, contradiction. So~$\varphi$ is in fact true.''
      \end{itemize}
      which constructively is only a proof of the weaker
      statement~$\neg\neg\varphi$, and
      \begin{itemize}
        \item a proof of a negated formula: ``Assume~$\psi$ were true. Then
        \ldots, contradiction. So~$\neg\psi$ holds.''
      \end{itemize}
      which is a perfectly fine in constructive mathematics.

      \item There is a similar rumor that constructive mathematicians \emph{deny}
      the law of excluded middle. In fact, one can constructively prove that there is no
      counterexample to the law: For any formula~$\varphi$, it holds
      that~$\neg\neg(\varphi \vee \neg\varphi)$.

      In constructive mathematics, one merely doesn't \emph{use} the law of
      excluded middle. (Only in concrete models, for example as provided by the
      internal universe of the sheaf topos on a non-discrete topological space,
      the law of excluded middle will actually be refutable.)

      \item See~\cite{troelstra:dalen:constr} for references about constructive
      mathematics.
    \end{itemize}
  }
}


\section{The petit Zariski topos of a scheme}
\subsection{Basic look and feel}

\frame[t]{\frametitle{The petit Zariski topos}
  \begin{block}{Definition}The \hil{petit Zariski topos} of a scheme~$X$ is the
  category~$\Sh(X)$ of set-valued sheaves on~$X$.\end{block}

  \begin{block}{Basic look and feel}
  \vspace{-0.5em}
  \begin{itemize}
    \item Internally, the structure sheaf~$\O_X$ looks like \[ \text{
    an ordinary ring.} \]
    \item Internally, a sheaf of~$\O_X$-modules looks like \[
    \text{an ordinary module on that ring.} \]
  \end{itemize}
  \end{block}
}

\subsection{Sheaves of modules}
\frame[t]{\frametitle{Sheaves of modules, internally}
  Let~$\F$ be a sheaf of~$\O_X$-modules.
  \begin{itemize}
    \item $\F$ is \hil{locally of finite type} iff internally, it is finitely
    generated:
    \begin{multline*}
      \Sh(X) \models \bigvee_{n\geq0} \exists x_1,\ldots,x_n\?\F\_
      \forall x\?\F\_ \exists a_1,\ldots,a_n\?\O_X\_
      x = \sum_i a_i x_i
    \end{multline*}

%    \item $\F$ is \hil{locally of finite presentation} iff internally,
%    it is finitely presented:
%    \[ \Sh(X) \models \speak{$\exists$ surjection~$\O_X^n
%    \twoheadrightarrow \F$ with f.\,g. kernel} \]

    \item $\F$ is \hil{locally finitely free} iff internally, it is a finitely free module:
    \[ \Sh(X) \models \bigvee_{n\geq0} \F \cong \O_X^n \]

%    \item $\F$ is \hil{coherent} iff internally, it is\ldots

    \item Similarly for finite presentation, coherence, flatness, \ldots
  \end{itemize}

  \vfill
  \hil{Motto:} We can understand notions of algebraic geometry as notions of
  linear algebra internal to~$\Sh(X)$.
}

\frame[t]{\frametitle{Proving in the internal universe}
  \begin{block}{Well-known proposition}
    Let $0 \lra \F' \lra \F \lra \F'' \lra 0$ be a short exact sequence
    of~$\O_X$-modules. If~$\F'$ and~$\F''$ are locally of finite type, so
    is~$\F$.
  \end{block}

  \vfill
  \begin{block}{Proof}
    Follows at once from the following theorem of constructive linear algebra:

    \begin{quote}
      Let~$0 \lra M' \lra M \lra M'' \lra 0$ be a short exact sequence of
      modules. If~$M'$ and~$M''$ are finitely generated, so is~$M$.
    \end{quote}
  \end{block}
  \vfill
%    \begin{enumerate}
%      \item From the internal point of view, the short exact sequence
%      is a sequence of ordinary modules with~$\F'$ and~$\F''$ being finitely
%      generated.
%
%      \item So~$\F$ is finitely generated too, by standard constructive linear algebra
%      -- interpreted internal to~$\Sh(X)$.
%
%      \item Externally, this means that~$\F$ is locally of finite type.
%    \end{enumerate}

  \hil{Motto:} We can understand basic statements of algebraic geometry as
  statements of linear algebra internal to~$\Sh(X)$.

  \note{
    \begin{itemize}
      \item Note that in the standard proof of the linear algebra fact, there
      is no actual \emph{choice} of generators happening (which would require
      the axiom of choice, which is not available in the internal universe):
      One simply uses the elimination rule for the existential quantifier.
    \end{itemize}
  }
}

\frame[t]{\frametitle{The sheaf of rational functions}
  \begin{block}{Classical definition}
  The sheaf~$K_X$ of \hil{rational functions} on a scheme~$X$ is the
  sheafification of
  \[ U \subseteq X \longmapsto \Gamma(U, \O_X)[S(U)^{-1}], \]
  where~$S(U) = \{ s \in \Gamma(U, \O_X) \,|\, \text{$s \in \O_{X,x}$ is
  regular for all~$x \in U$} \}$.
  \end{block}

  \begin{block}{Internal definition}
  $K_X$ is the total quotient ring of~$\O_X$.
  \end{block}
}

\subsection{The structure sheaf in more detail}
\frame[t]{\frametitle{The structure sheaf, internally}
  \begin{itemize}
    \item Internally, $\O_X$ is a local ring: \[ \Sh(X) \models \forall x,y\?\O_X\_
      \speak{$x+y$ inv.} \Longrightarrow
      \speak{$x$ inv. or $y$ inv.} \]

    \item In fact, $\O_X$ is ``almost'' a field:
    \[ \Sh(X) \models \forall x\?\O_X\_
      \neg\,\speak{$x$ inv.} \Longrightarrow \speak{$x$ is nilpotent} \]
    \vfill

    \item The scheme $X$ is reduced iff $\O_X$ is internally a reduced ring,
    \[ \Sh(X) \models \bigwedge_{n \geq 0} \forall x\?\O_X\_ x^n = 0 \Longrightarrow
    x = 0, \]
    iff $\O_X$ fulfills the field condition
    \[ \Sh(X) \models \forall x\?\O_X\_
      \neg\,\speak{$x$ inv.} \Longrightarrow x = 0. \]
  \end{itemize}

  \note{
    \begin{itemize}
      \item The usual definition of a local ring (a nontrivial ring with
      exactly one maximal ideal) is equivalent to the more elementary condition
      given here -- in classical logic. In constructive logic, the latter is
      much better behaved. (For instance, recall that one even needs Zorn's
      lemma to show the existence of a maximal ideal in a nontrivial ring.)

      \item In constructive mathematics, the notion of \emph{field} bifurcates
      into several related but non-equivalent ones. A non-comprehensive list
      is the following:
      \begin{align*}
        \forall x\_ & x = 0 \vee \speak{$x$ inv.} \\
        \forall x\_ & \neg\,\speak{$x$ inv.} \Rightarrow x = 0 \\
        \forall x\_ & x \neq 0 \Rightarrow \speak{$x$ inv.}
      \end{align*}
    \end{itemize}
  }
}

\subsection{On the rank of sheaves of modules}
\frame[t]{\frametitle{Rank of a sheaf of modules}
  \hspace{-1em}Let~$\F$ be an~$\O_X$-module locally of finite type.
  There is a correspondence
%  \begin{columns}[t,onlytextwidth]
%    \hspace{-1em}
%    \begin{column}{0.5\textwidth}
%      \begin{block}{Classical definition}
%        \ \\[-0.5em]
%        The \hil{rank function} of~$\F$ is
%        \[ \begin{array}{@{}rcl@{}}
%          X &\longrightarrow& \NN \\
%          x &\longmapsto& \dim_{k(x)} \F_x / \mathfrak{m}_x \F_x.
%        \end{array} \]
%      \end{block}
%    \end{column}
%    \hspace{2em}
%    \begin{column}{0.5\textwidth}
%      \begin{block}{Internal definition}
%        \ \\[-0.5em]
%        The \hil{rank} of~$\F$ is the
%        minimal number of generators of~$\F$,
%        considered as an element of the \hil{completed natural numbers}.
%      \end{block}
%    \end{column}
%  \end{columns}
  \[
    \underunbrace{\fmini{0.35\textwidth}{upper semi-continuous \\ functions on~$X$}}_{\text{external}}
    \xleftrightarrow{\text{1-1 correspondence}}
    \underunbrace{\fmini{0.4\textwidth}{completed natural numbers}}_{\text{internal}}
  \]
  \hspace{-1em}under which the rank function of~$\F$ maps to the \hil{minimal
  number of \\\hspace{-1em}generators of~$\F$}.

  \vfill

  \begin{columns}[t,onlytextwidth]
    \hspace{-1em}
    \begin{column}{0.5\textwidth}
      \begin{block}{Well-known proposition}
        \ \\[-1.1em]
        Assume~$X$ to be reduced. \\ Then:
        \begin{enumerate}
        \item $\F$ is locally free iff its rank is locally constant.
        \item $\F$ is locally free on a dense open subset.
        \end{enumerate}
      \end{block}
    \end{column}
    \hspace{2em}
    \begin{column}{0.5\textwidth}
      \begin{block}{Constructive linear algebra}
        \ \\[-1.1em]
        Let~$M$ be a f.\,g. module over a field. Then:
        \begin{enumerate}
          \item $M$ is free iff above number is an ordinary natural number.
          \item $M$ is always not not free.
        \end{enumerate}
      \end{block}
    \end{column}
  \end{columns}

  \note{
    \begin{block}{Proposition}
      If every inhabited subset of the natural numbers has a minimum, then the
      law of excluded middle holds.
    \end{block}

    \begin{block}{Proof}
      Let~$\varphi$ be an arbitrary formula. Define the subset
      \[ U := \{ n \in \NN \,|\, n = 1 \vee \varphi \} \subseteq \NN, \]
      which surely is inhabited by~$1 \in U$. So by assumption, there exists a
      number~$z \in \NN$ which is the minimum of~$U$. We have
      \[ z = 0 \quad\vee\quad z > 0 \]
      (this is not obvious, but can be proven by induction).

      If~$z = 0$, we have~$0 \in U$, so~$0 = 1 \vee \varphi$, so~$\varphi$
      holds.

      If~$z > 0$, then~$\neg\varphi$ holds: Because if~$\varphi$ were
      true, zero would be an element of~$U$, contradicting the minimality of~$z$.
    \end{block}
  }

%  \note{
%    \begin{block}{Proposition}
%      If every finitely generated module over a field admits a basis, then the
%      law of excluded middle hods.
%    \end{block}
%
%    \begin{block}{Proof}
%      Let~$\varphi$ be an arb
  % XXX: Stimmt das?
}

\note{
  \begin{block}{Proposition}
    The partially ordered set
    \[ \widehat\NN := \{ A \subseteq \NN \,|\, \text{$A$ inhabited and upward
    closed}
    \} \]
    is the least partially ordered set containing~$\NN$ and possessing minima
    of arbitrary inhabited subsets.
  \end{block}

  \begin{block}{External translation (see~\cite{mulvey:intalg})}
    Let~$X$ be a topological space and consider the constant sheaf~$N$ with
    $\Gamma(U, N) = \{ f : U \to \NN \,|\ \text{$f$ continuous} \}$.
    Then there is a 1-1 correspondence:
    \begin{enumerate}
      \item Let~$A \hookrightarrow N$ be a subobject which is inhabited
      and upward closed from the internal point of view. Then
      \[
        x \longmapsto \inf\{ n \in \NN \,|\, n \in A_x \}
      \]
      is an upper semi-continous function on~$X$.

      \item Let~$\alpha : X \to \NN$ be a upper semi-continous function. Then
      \[ U \subseteq X \longmapsto \{ f : U \to \NN \,|\, \text{$f$
      continuous,\ \ $f \geq \alpha$ on~$U$} \} \]
      is a subobject of~$N$ which internally is inhabited and upward closed.
    \end{enumerate}
  \end{block}
}

\note{
  \begin{itemize}
    \item There is the so-called \emph{negative translation}: To a zeroth
    approximation, this reads that a formula~$\varphi$ is true classically
    iff~$\neg\neg\varphi$ is true constructively. (Beware that this is a
    oversimplification! You can find the exact statement
    in~\cite{coquand:negative}.)
  \end{itemize}
}


\section{The gros Zariski topos of a scheme}
\subsection{Basic look and feel}
\frame[t]{\frametitle{The gros Zariski topos}
  \begin{block}{Definition}The \hil{gros Zariski topos $\Zar(S)$} of a scheme~$S$ is the
  category $\Sh(\Sch/S)$,
  i.\,e. consists of certain functors
  $(\Sch/S)^\op \longrightarrow \Set$.
  \end{block}

  \begin{block}{Basic look and feel}
  \vspace{-0.5em}
  \begin{itemize}
    \item For each~$S$-scheme~$X$, its functor of points~$\ul{X}$ is an object
    of~$\Zar(S)$. It feels like
    \[ \text{the set of points of~$X$}. \]
    \item Internally, $\affl$ (affine line), given by
    \[ \affl : X \longmapsto \Gamma(X, \O_X), \]
    looks like a field:\ \ $\Zar(S) \models \forall x\?\affl\_ x \neq 0
    \Longrightarrow \speak{$x$ inv.}$
  \end{itemize}
  \end{block}

  \note{
    \begin{itemize}
      \item The overcategory~$\Sch/S$ becomes a Grothendieck site by declaring
      families of jointly surjective open immersions to be covers. See for
      instance the excellent Stacks project~\cite{stacks} for details.

      \item Hakim worked out a theory of schemes internal to toposes (but
      without using the internal language), see~\cite{hakim:rel}.

      \item The internal language of~$\Zar(\Spec A)$ is related to Coquand's
      program about dynamical methods in algebra,
      see~\cite{coquand:completeness,coquand:logical,coquand:effective}.

      \item The observation that~$\affl$ is internally a field is due to
      Kock~\cite{kock:univproj} (in the case~$S = \Spec\ZZ$).
    \end{itemize}
  }
}

\note{
  \begin{itemize}
    \item The affine line fulfills the axiom
      \[ \Zar(S) \models \speak{every function~$\affl \to \affl$ is a
      polynomial}. \]

      Compare with the axiom of synthetic differential geometry:
      \[ \text{every function~$\RR \to \RR$ is smooth.} \]

      See~\cite{kock:sdg}.
  \end{itemize}
}


\subsection{Group schemes}
\frame[t]{\frametitle{Group schemes}
  \vspace{1em}
  \hspace{-1em}
  \begin{tabular}{l|l|l}
    group scheme & internal definition & functor of points: $X \mapsto \ldots$ \\[0.5em]\hline
    && \\[-0.5em]
    $\GG_\text{a}$ & $\affl$ (as additive group) & $\Gamma(X, \O_X)$ \\[0.5em]
    $\GG_\text{m}$ & $\{ x \? \affl \,|\, \text{$x$ inv.} \}$ & $\Gamma(X, \O_X)^\times$ \\[0.5em]
    $\mu_n$ & $\{ x \? \affl \,|\, x^n = 1 \}$ & $\{ f \in \Gamma(X, \O_X) \,|\, f^n = 1 \}$ \\[0.5em]
    $\GL_n$ & $\{ M \? \affl^{n \times n} \,|\, \text{$M$ inv.} \}$ & $\GL_n(\Gamma(X, \O_X))$
  \end{tabular}
  \vspace{2em}

  \hil{Motto:} Internal to~$\Zar(S)$, group schemes look like ordinary groups.
}


\subsection{Open immersions}
\frame[t]{\frametitle{Open immersions}
  \begin{block}{Definition}
    A formula~$\varphi$ is called
    \begin{itemize}
      \item \hil{decidable} iff~$\varphi \vee \neg\varphi$ holds.
      \item \hil{$\neg\neg$-stable} iff~$\neg\neg\varphi \Rightarrow \varphi$
      holds.
    \end{itemize}
  \end{block}

  We can similarly define a notion of~\hil{open} formulas in~$\Zar(S)$.

  \begin{block}{Proposition}
    \vspace{-1em}
    \begin{itemize}
      \item For~$f \in \Gamma(X,\O_X)$, the formula~``$f(x) \neq 0$'' is open.
      \item A morphism~$p : Y \to X$ is an open immersion iff
            \[ \Zar(S) \models \speak{$\ul{p}$ is injective and ``$x \in
            \operatorname{image}(\ul{p})$'' is open}. \]
    \end{itemize}
  \end{block}
}


\subsection{Open problems}
\frame[t]{\frametitle{Open problems}
  \begin{itemize}
  \item
  Find internal characterizations for morphisms of schemes
  \begin{itemize}
    \item being of finite type,
    \item being separated,
    \item being proper (related work:~\cite{moerdijk:vermeulen:proper}),
    \item \ldots
  \end{itemize}

  \item
  Find interesting and useful properties of the internal language of specific
  schemes, for instance projective~$n$-space.

  \item Find general transfer principles for statements about~$A$-modules~$M$
  vs.~$\O_{\Spec A}$-modules~$M^\sim$.

  \item Contemplate about using derived methods in the internal setting.
  \end{itemize}
}

\section*{Conclusion}
\frame[t]{\frametitle{Conclusion}
  \begin{center}
    \hil{\large Usual mathematics is\ldots}
  \end{center}
  \begin{columns}[onlytextwidth]
    \begin{column}{0.5\textwidth}
      formal manipulation of symbols
    \end{column}
    vs.
    \hspace{0.5em}
    \begin{column}{0.5\textwidth}
      discovering exciting relationships
    \end{column}
  \end{columns}

  \begin{center}
    \hil{\large Mathematics internal to a topos is\ldots}
  \end{center}
  \begin{columns}[onlytextwidth]
    \begin{column}{0.5\textwidth}
      a technical device for simplifying routine proofs
    \end{column}
    vs.
    \hspace{0.5em}
    \begin{column}{0.5\textwidth}
      exploring exciting alternative universes
      and their curious objects
    \end{column}
  \end{columns}

  \vfill
  \begin{center}
    \small
    annotated version of these slides: \\
    \url{http://xrl.us/gaeltopos}
  \end{center}
%  \vfill
%  \begin{block}{Related work}
%    \begin{itemize}
%      \item Synthetic differential geometry: Kock, Moerdijk, \ldots
%      \item Constructive mathematics
%      \item Homotopy type theory: Voevodsky, \ldots
%    \end{itemize}
%  \end{block}
}

\backupstart
\frame[t,allowframebreaks]{\frametitle{References}
  \begin{thebibliography}{18}
    \setbeamertemplate{bibliography item}[text]
    %\beamertemplatebookbibitems
    \bibitem{caramello:bridges}
      O. Caramello.
      \newblock {\em The unification of Mathematics via Topos Theory}.
      \newblock arXiv 1006.3930, 2010.

    %\beamertemplatearticlebibitems
    \bibitem{coquand:negative}
      T. Coquand.
      \newblock Computational content of classical logic.
      \newblock \emph{Semantics and Logics of Computation}, 33--78, 1997.

    %\beamertemplatearticlebibitems
    \bibitem{coquand:completeness}
      T. Coquand.
      \newblock A completeness proof for geometrical logic.
      \newblock \emph{Logic, Methodology and Philosophy of Sciences}, 79--90, 2005.

    %\beamertemplatearticlebibitems
    \bibitem{coquand:logical}
      T. Coquand and H. Lombardi.
      \newblock A logical approach to abstract algebra.
      \newblock \emph{Math. Structures Comput. Sci}, 16(5):885--900, 2006.

    %\beamertemplatearticlebibitems
    \bibitem{coquand:effective}
      M. Coste, H. Lombardi and M. F. Roy.
      \newblock Dynamical methods in algebra: effective Nullstellensätze.
      \newblock \emph{Ann. Pure Appl. Logic}, 111(3):203--256, 2001.

    %\beamertemplatebookbibitems
    \bibitem{chapman:rowbottom:relcat}
      J. Chapman and F. Rowbottom.
      \newblock {\em Relative Category Theory and Geometric Morphisms: A Logical Approach}.
      \newblock Clarendon Press, 1992.

    %\beamertemplatebookbibitems
    \bibitem{hakim:rel}
      M. Hakim.
      \newblock {\em Topos annelés et schémas relatifs}.
      \newblock Springer-Verlag, 1972.

    %\beamertemplatebookbibitems
    \bibitem{johnstone:elephant}
      P. T. Johnstone.
      \newblock {\em Sketches of an Elephant: A Topos Theory Compendium}.
      \newblock Oxford University Press, 2002.

    %\beamertemplatearticlebibitems
    \bibitem{kock:univproj}
      A. Kock.
      \newblock Universal projective geometry via topos theory.
      \newblock \emph{J. Pure Appl. Algebra}, 9(1):1--24, 1976.

    %\beamertemplatebookbibitems
    \bibitem{kock:sdg}
      A. Kock.
      \newblock {\em Synthetic Differential Geometry}.
      \newblock Cambridge University Press, 2006.

    %\beamertemplatebookbibitems
    \bibitem{moerdijk:maclane:sheaves}
      S. Mac Lane and I. Moerdijk.
      \newblock {\em Sheaves in Geometry and Logic: a First Introduction to Topos Theory}.
      \newblock Springer-Verlag, 1992.

    %\beamertemplatebookbibitems
    \bibitem{moerdijk:vermeulen:proper}
      I. Moerdijk and J. J. C. Vermeulen.
      \newblock {\em Proper maps of toposes}.
      \newblock American Mathematical Society, 2000.

    %\beamertemplatearticlebibitems
    \bibitem{mulvey:intalg}
      C. Mulvey.
      \newblock Intuitionistic algebra and representations of rings.
      \newblock \emph{Mem. Amer. Math. Soc.}, 148:3--57, 1974.

    %\beamertemplatearticlebibitems
    \bibitem{shulman:stack}
      M. Shulman.
      \newblock Stack semantics and the comparison of material and structural set theories.
      \newblock \emph{arXiv} 1004.3802, 2010.

    %\beamertemplatebookbibitems
    \bibitem{stacks}
      The Stacks Project Authors.
      \newblock {\em Stacks Project}.
      \newblock 2013.

    %\beamertemplatebookbibitems
    \bibitem{troelstra:dalen:constr}
      A. S. Troelstra and D. van Dalen.
      \newblock {\em Constructivism in Mathematics: An Introduction}.
      \newblock North-Holland Publishing, 1988.

    %\beamertemplatearticlebibitems
    \bibitem{wraith:int}
      G. C. Wraith.
      \newblock Intuitionistic algebra: some recent developments in topos theory.
      \newblock \emph{Proceedings of ICM}, 1978.

    %\beamertemplatearticlebibitems
    \bibitem{wraith:galois}
      G. C. Wraith.
      \newblock Generic Galois theory of local rings.
      \newblock \emph{Lecture Notes in Math.}, 753:739--767, 1979.
  \end{thebibliography}
}
\backupend


% es fehlt nur noch:
% quasicompactness
% qcoh modules

% XXX: Vereinfachungsregeln

\end{document}
